\documentclass[12pt]{article}
\usepackage{graphicx}

\begin{document}

\title{Reverse Engineering CAN 101}
\author{Bhargava Manja}
\maketitle

\textit{Disclaimer: This is more an art than a science}

First, I will explain how CAN works for different parts of the car in different
regimes to help you reason about CAN, then detail my techniques for reverse
engineering.

\section{CAN}
\subsection{Introduction}
Electronic control units (ECUs) control most functions of modern vehicles. They
are small embedded computers that all talk to one another with a controller
area network (CAN), sort of like a LAN. The difference between LAN networking
and CAN networking is that all CAN devices are connected to all other CAN devices
because they are all connected to a common CAN bus. ECUs communicate over this
bus via CAN frames, the CAN analogue to an ethernet packet. If every ECU can
hear all the talk over the CAN bus, how does an ECU know which frames are meant
for it to listen and respond to? Each CAN frame begins with an id that uniquely
identifies the ECU that is meant to recieve the frame. Each frame also has
a body, which contains data for the recipient. 

\subsection{Frames}
There are two kinds of frames: standard and extended. The only difference
between the two is that extended frames have longer ids and more room for data.
All ECUs on a bus must be using the same kind of frame. I have wasted hours not
being able to communicate over a bus before realizing I am sending extended
frames on a standard CAN network.

\subsection{Frequency}
ECUs are configured to communicate over the CAN bus via a specific frequency.
That is, if I have one CAN bus with two ECUs, they must be configured to
communicate with the same frequency. You can think of frequency like language
in that two ECUs on two different frequencies can't understand one another.

\section{Grigore's car}
Grigore's car actually contains 2 CAN networks. One is the body CAN, or BCAN,
over which most components of the car communicate. For example, the headlights
and windshield wiper ECUs are on the BCAN. They work like so:
\begin{enumerate}
\item{when a driver turns the blinker on, they change the state on a control
   ECU under the steering wheel}
\item{the control ECU sends a stream of messages to the headlight ECU. These 
   messages each say the same thing, which amounts to "turn left blinker on" 
   (for example). This sort of message is a \textbf{stream message}. If only one is
   sent, the blinker will only turn on for a fraction of a second. The control
   ECU must continue to send messages for the blinker to continue to blink.}
\item{If the driver turns off the blinker and turns on high beams, the control ECU
   ends the first stream and
   sends another stream to the same headlight ECU, this time with a different
   data field indicating a different action (high beams vs blinkers).}
\end{enumerate}
The BCAN operates at 125 kbps, and uses extended frames.

The other network is the fast CAN, or FCAN. FCAN connects components that can
not be obstructed, and must operate quickly. For example, FCAN connects your
pedal to the engine. FCAN on Grigore's car operates at 500 kbps and uses
standard frames. 

\subsection{Atomic Messages}
The door open and close functionality does not use stream messages, but uses
\textbf{atomic messages}. That is, one message is sufficient to open or close the
door. There may be other such exceptions, but this is the only one so far.
The door open/close buttons on the driver dash are actually wired straight to
the motors of the car and don't use CAN. The key open/close buttons send
a radio signal to an ECU that always stays on and sends a single open/close
message to the door ECU

\section{Reverse Engineering}
To reverse engineer a function of the car, you need to find out \textbf{the ID of the
ECU that controls that function} and the \textbf{data payload to trigger that function}. 
Then you can send a frame over the bus with the proper ID and payload to
activate that function. To determine both of these, I do the following:
\begin{enumerate}
\item{Connect the USB-CANII tool to a CAN splice}
\item{Start the CANDebug software tool and record the target vehicle function's
CAN trace}
\item{Run the FrameDecoder tool to find candidate ECU ids for further
inspection}
\item{Use CANDebug's \textit{display filter} and \textit{summary display}
capabilities to winnow down the list of possible ECU ids}
\end{enumerate}

I assume you can do number 1, and will explain the rest below:
\subsection{Recording Traces via CANDebug}
First, click "Save Real Time" and choose a file name. Then click "Start" and 
choose the appropriate parameter for baudrate, and click "Ok".
\end{figure}

Perform the target function (e.g turn the windshield wipers on or clock the
door lock button on the key). Through some trail and error I have determined
the following heuristic best practices:
\begin{itemize}
\item{Do not collect target logs for longer than 10 seconds. I usually stick to
between 3 and 5}
\item{record multiple target logs}
\item{Before each recording, I turn the car off, turn it on (ignition), wait
for a few seconds, and begin recording}
\end{itemize}

\subsection{Finding Candidate ECUs}
Before you continue, open up your logs and delete the first 3 lines of each.
They are gunk that will cause the FrameDecoder tool to fail. The help text of
FrameDecoder:
\begin{verbatim}
  -h, --help            show this help message and exit
  -c, --code-gen        generate C code for STM32 boards
  -i IDLE_LOG, --idle-log=IDLE_LOG
                        contains frames generated when car is idling
  -a ACTION_LOG, --action-log=ACTION_LOG
                        contains frames generated when car does action we want
                        to reverse engineer
  -r IGNORED_IDS, --ignore=IGNORED_IDS
                        comma separated list of message IDs toignore when
                        generating diff
\end{verbatim}
This tool takes an \textit{idle log} (a CAN trace of the car idling) and an
\textit{action log} (a log of the target action taking place, what you have
just recorded) and returns a list of CAN frames that are in the action log but
are not in the idle log, in the order in which they occur. This is a candidate
list of ECUs that might control the target function. 

Now, we must pare this list down. First, reconnect to the vehicle. 


\end{document}
